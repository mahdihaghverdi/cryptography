\documentclass{article}

% URLs and hyperlinks ---------------------------------------
\usepackage{hyperref}
\hypersetup{
    colorlinks=true,
    linkcolor=blue,
    filecolor=magenta,      
    urlcolor=blue,
}
\usepackage{xurl}
%----------------------------------------------------

\usepackage{amsmath, amssymb, empheq}
\usepackage{enumitem}
\usepackage{adjustbox}
\usepackage{euler}
\usepackage{graphicx}
\usepackage{float}

\usepackage{xepersian}
\settextfont{Yas}
\newcommand{\modts}{\overset{26}{\equiv}}

\title{تکلیف دوم}
\author{مهدی حق‌وردی}
\date{\today}

\begin{document}
\maketitle

\section{}
\textbf{{\large با توجه به سیستم رمزنگاری \lr{DES} به سوالات زیر پاسخ دهید.}}

\begin{enumerate}[label=\alph*)]
\item 
تعداد کل عملیات‌های 
\lr{\texttt{xor}}
را بدست آورید.

\item 
هدف از 
\lr{s-box}ها
را بنویسید.

\item 
پیچیدگی حمله‌ی جست‌وجوی جامع به این سیستم از چه مرتبه‌ای می‌باشد؟

\item 
 در چه حمله‌ای به این سیستم پیچیدگی زمانی از مرتبه‌ی
$2^{56}$
خواهد شد؟

\item 
اگر خروجی سیستم رمزنگاری به یک سیستم رمزنگاری دیگر داده شود، چه تغییری در
امنیت آن حاصل می‌شود؟ \lr{(double des)} اگر این کار سه بار تکرار شود
چطور؟ 
\lr{(triple des)}

\item 
ویژگی مکمل بودن این سیستم را ثابت کنید و توضیح دهید در آن صورت حمله به
این سیستم از چه مرتبه‌ایست و چرا؟

خاصیت مکمل بودن 
\lr{DES}:
\begin{latin}
\begin{equation}
\text{DES}_\text{K}(\text{M}) = \text{C}
\Rightarrow
\text{DES}_{\overline{\text{K}}}(\overline{\text{M}}) = \overline{\text{C}}
\end{equation}
\end{latin}
\end{enumerate}

\section{}
{\large \textbf{با استفاده از یک کلید رمز واحد، هر یک ازتبدیالت زیر را بر متن آشکار که تنها در بیت اول با هم تفاوت دارند، اعمال کنید. تعداد بیت‌های تغییر یافته پس از هرتبدیل را پیدا کنید. هرتبدیل را بطور مستقل اعمال کنید. در مورد اثر بهمنی پس از هرتبدیل بطور مستقل و سپس اثر بهمنی پس از اعمال یک راند توضیح دهید.}}

\begin{enumerate}[label=\alph*)]
\item \lr{subBytes}
\item \lr{shiftRows}
\item \lr{mixColumns}
\item \lr{addRoundKey}
\end{enumerate}

\section{}
{\large \textbf{از بین مدهای عملیاتی
\lr{ECB},
\lr{CBC},
\lr{OFB},
\lr{CFB}
و
\lr{CTR}
در کدام یک امکان افزایش سرعت در عمل رمزگذاری با استفاده از
\lr{parallel processing}
یا پردازش موازی وجود دارد؟}}
\end{document}