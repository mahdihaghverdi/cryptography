\documentclass{article}
% URLs and hyperlinks ---------------------------------------
\usepackage{hyperref}
\hypersetup{
    colorlinks=true,
    linkcolor=blue,
    filecolor=magenta,      
    urlcolor=blue,
}
\usepackage{xurl}
%----------------------------------------------------

\usepackage{amsmath}
\usepackage{xepersian}
\settextfont{Yas}

\title{تکلیف چهارم}
\author{مهدی حق‌وردی}

\begin{document}
\maketitle
\tableofcontents

\section{دیفی-هلمن سه نفره}
\begin{latin}
One possible protocol could be the following:

\begin{enumerate}
\item 
A, B, C each generate their private keys $x_A$, $x_B$, $x_C$

\item 
A, B, C each calculate $y_A = g^{x_A}$, $y_B = g^{x_B}$, $y_C = g^{x_C}$

\item 
A sends $y_A$ to B, B sends $y_B$ to C, C sends $y_C$ to A.

\item 
A calculates $z_{CA} = y^{x_A}_{C}$, B calculates $z_{AB} = y^{x_B}_{A}$, C calculates $z_{BC} = y^{x_C}_{B}$.

\item 
A sends $z_{CA}$ to B, B sends $z_{AB}$ to C, C sends $z_{BC}$ to A.

\item 
A calculates $k_{BCA} = z^{x_A}_{BC}$, 
B calculates $k_{CAB} = z^{x_B}_{CA}$,
C calculates $k_{ABC} = z^{x_C}_{AB}$.
\end{enumerate}

The above equality means that the three parties now know a common secret $k_{ABC} = k_{CAB} = k_{BCA}$

\end{latin}
\section{رمزنگاری بهینه در برقراری یک نشست \lr{(session)}}

\section{معکوس ضربی}
\begin{latin}
Find $19^{-1}  \mod 999$ using EEA.

\begin{itemize}
\item $999 \overset{999}{\equiv} 0 \times 19$
\item $19 \overset{999}{\equiv} 1 \times 19$
\item $11 = 999 - (52 \times 19) \overset{999}{\equiv} -52 \times 19$
\item $8 = 19 - (1 \times 11) \overset{999}{\equiv} (1\times19) - (-52\times19) = 53\times19$

\item $3 = 11 - (1\times8) \overset{999}{\equiv}
(-52\times19)-(53\times19) = -105\times19$

\item $2 = 8 - (2\times3) \overset{999}{\equiv}
(53\times19) - (2\times(-105\times19)) = 263\times19$

\item $1 = 3 - (1\times2) \overset{999}{\equiv}
(-105\times19)-(263\times19) = -368\times19$

\item $\Rightarrow -368 \mod 999 = 631$ \texttt{<-- answer}

\item checking the answer

$19 \times 631 = 11989 \mod 999 = 1$

\end{itemize}
\end{latin}

\newpage
\section{در \lr{RSA}}
\subsection{پیدا کردن \lr{d}}
\begin{latin}
\begin{itemize}
\item[] $d = 17^{-1} \mod \Phi(3937)$
\item[] $\Phi(3937) = \Phi(31\times127) = 30\times126 = 3780$
\item[] $3780 \overset{3780}{\equiv} 0 \times 17$
\item[] $17 \overset{3780}{\equiv} 1 \times 17$
\item[] $6  = 3780 - (222\times17) \overset{3780}{\equiv} -222 \times 17$
\item[] $5  = 17 - (2\times6) \overset{3780}{\equiv}
(1\times17) - (2\times(-222\times17)) = 445\times17$
\item[] $1  = 6 - (1\times5) \overset{3780}{\equiv}
(-222\times17) - (445\times17)  = -667 \times 17$
\item[] $-667 \mod 3780 = 3113$  \texttt{<-- d}
\item[] check the answer

$17 \times 3113 = 52921 \mod 3780 = 1$
\end{itemize}
\end{latin}

\subsection{پیدا کردن \lr{n}, \lr{d} و \lr{$\Phi(n)$}}
\begin{latin}
\begin{itemize}
\item[] $n = pq = 17\times23 = 391$ \texttt{<-- n}
\item[] $\Phi(n) = (p-1)(q-1) = 352$ \texttt{<-- $\Phi(n)$}
\item[] $d = 3^{-1} \mod \Phi(n)$
\item[] $352 \overset{352}{\equiv} = 0\times3$
\item[] $3 \overset{352}{\equiv} = 1\times3$
\item[] $1 = 352 - (117*3) \overset{352}{\equiv} =
-117 \times 3$
\item[] $-117 \mod 352 = 235$ \texttt{<-- d}

\item[] check the answer

$235\times3 = 705 \mod 352 = 1$
\end{itemize}
\end{latin}


\subsection{چرا \lr{e} را عدد یک انتخاب نمی‌کنیم؟}
برای اینکه در هر مجموعه‌ی 
$Z^*_n$
عی، معکوس ۱ می‌شود ۱.

\subsection{حمله‌ی \lr{chosen-ciphertext} روی \lr{RSA}}
چون این فرد متن رمزشده (\lr{c = 57}) و اطلاعات کلید عمومی (\lr{e} و \lr{n}) را دارد و در این مثال \textbf{مقدار \lr{n} کوچک است} میتواند آن را تجزیه کند و سپس \lr{$\Phi$} را محاسبه کند و با مقادیر مختلف کلیدی که برای \lr{decryption} انتخاب و آزمون و خطا می‌کند به \lr{p} برسد.\LTRfootnote{\url{https://www.geeksforgeeks.org/chosen-ciphertext-attacks-on-rsa/}}
\subsection{آیا کلید \lr{regenrate} شده امن است؟}
از نظر من جفت کلید جدید امن نیستند. به این دلیل که مهاجم اکنون از معادله‌ی 
\lr{$d = e^{-1} \mod \Phi(n)$}
تنها یک مجهول دارد که (چون مطمئن نیستم روش ریاضی دارد یا خیر) می‌تواند با آزمون و خطا هم به فی دست پیدا کرده و از روی کلید عمومی جدید براحتی کلید خصوصی جدید را محاسبه کند.

\section{در \lr{Rabin}}
\subsection{متن ۱۷ را رمز کنید}
\begin{latin}
\begin{itemize}
\item[] $m^2 \mod \Phi(n)$
\item[] $n = 47\times11 = 517$
\item[] $\Phi(n) = 46\times10 = 460$
\item[] $c = 17^2 \mod 460 = 289$  \texttt{<-- ciphertext}
\end{itemize}
\end{latin}

\subsection{با استفاده از \lr{Chinese remainder theorem} چهار متن آشکار احتمالی را پیدا کنید}

\begin{latin}
\begin{equation*}
\sqrt{c} \mod n = \left[q\times(\pm c^{\frac{p+1}{4}})\underbrace{(q^{-1} \mod p)}_{5}\right] + \left[p\times(\pm c^{\frac{q+1}{4}})\underbrace{(p^{-1} \mod q)}_{4}\right]
\end{equation*}

\begin{enumerate}
\item $(11\times289^{12}\times5) + (47\times289^{3}\times4)$
\item $(11\times289^{12}\times5) + (47\times-289^{3}\times4)$
\item $(11\times-289^{12}\times5) + (47\times289^{3}\times4)$
\item $(11\times-289^{12}\times5) + (47\times-289^{3}\times4)$
\end{enumerate}
\end{latin}
\section{امنیت امضای دیجیتال \lr{RSA}}

\section{سوء‌استفاده‌ی فرد مهاجم از روی ویژگی همریختی \lr{RSA}}

\section{سختی جعل در امضای الجمال}

\end{document}